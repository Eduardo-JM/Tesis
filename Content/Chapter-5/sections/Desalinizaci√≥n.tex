\section{Desalinización}
	
	\subsection{Métodos de desalinización}
		
		Los métodos de desalinización se pueden dividir en 4 categorías según el principio de trabajo; como se observa en la~\cref{fig:métodos-desalinización}, se pueden agrupar en:
		
		\begin{itemize}
			\item \textbf{Métodos térmicos:} Con el conocimiento de los puntos de evaporación y congelación del agua y las condiciones de operación que se tendrán, se establecen técnicas que permiten extraer las sales al modificar la temperatura.
			\item \textbf{Métodos de membrana:} Con fuerzas impulsoras sean mecánicas o eléctricas, se usa una barrera física (membrana) con poros de tamaño específico para separar las sales del agua atrapándolas en los poros.
			\item \textbf{Métodos químicos:} Se hace uso de propiedades químicas como diferencia de solubilidad, transporte de iones disueltos, mecanismos de nucleación, entre otros para aprovechar estos fenómenos para purificar solventes, entre ellos, el agua.
			\item \textbf{Adsorción:} Con el descubrimiento de nuevos materiales se usan materiales porosos adsorbentes que purifican el agua; posteriormente se les inyecta otra cantidad de energía para limpiarlos y así poder ser utilizados de nuevo.
		\end{itemize}
		
		\begingroup
			\tikzexternaldisable
			\begin{figure}[H]
				\centering
				\begin{forest}
	for tree={
		line width=0.5pt,
		draw=linecol,
		rect,
		calign=edge midpoint,
		minimum size=25pt,
		text width=50mm,
		rounded corners=5pt,
		child anchor=north,
		parent anchor=south,
		drop shadow,
		l sep+=12.5pt,
		inner color=background,
		anchor=center,
		edge path={
			\noexpand\path[color=linecol, rounded corners=5pt,>={Stealth[length=10pt]}, line width=0.5pt, ->, \forestoption{edge}]
			(!u.parent anchor) -- +(0,-5pt) -|
			(.child anchor)\forestoption{edge label};
		},
		if = {level == 1}{outer color=lightblue, text width=38mm}{},
		if = {level == 2}{
			outer color=lightgreen,
			for tree = {
				text width=28mm,
				child anchor=west
			}
		}{},
		if = {level == 3}{outer color=lightpink, text width=12mm}{}
	}
	[Clasificación, outer color=lilac,
	for tree = {
		for children = {
			grow'=0,
			folder
		}
	}
		[Térmicos
			[Adición
				[\acrshort{msf}]
				[\acrshort{med}]
				[\acrshort{mvc}]
				[\acrshort{hdh}]
				[\acrshort{ds}]
			]
			[Extracción
				[\acrshort{frz}]
			]
		]
		[Membrana
			[\acrshort{ro}]
			[\acrshort{fo}]
			[\acrshort{ed}]
			[\acrshort{nf}]
		]
		[Químicos
			[\acrshort{iex}]
			[\acrshort{lle}]
			[\acrshort{ghyd}]
			[Otros]
		]
		[Adsorción]
	]
\end{forest}

				\caption{Esquematización de los diferentes métodos de \gls{desalinizacion}}
				\floatfoot{Elaboración propia}
				\label{fig:métodos-desalinización}
			\end{figure}
			\tikzexternalenable
		\endgroup
	
	\subsection{Destilación solar}
		
		Los destiladores solares caen dentro de los métodos térmicos aditivos de la desalinización, a su vez, estos se pueden agrupar en destiladores solares activos y pasivos. Los \acrshort{ds} activos son aquellos que incorporan una fuente de energía externa para aumentar la producción del sistema; entre los mecanismos que se pueden incorporar existen concentradores solares, calor residual de procesos industriales, calentadores eléctricos o de combustible, etc. En la
		
		\begin{figure}[H]
			\centering
			\begin{forest}
	for tree={
		line width=0.5pt,
		draw=linecol,
		rect,
		calign=edge midpoint,
		minimum size=25pt,
		text width=75mm,
		rounded corners=5pt,
		child anchor=north,
		parent anchor=south,
		drop shadow,
		l sep+=12.5pt,
		inner color=background,
		anchor=center,
		edge path={
			\noexpand\path[color=linecol, rounded corners=5pt,>={Stealth[length=10pt]}, line width=0.5pt, ->, \forestoption{edge}]
			(!u.parent anchor) -- +(0,-5pt) -|
			(.child anchor)\forestoption{edge label};
		},
		if = {level == 1}{outer color=lightblue, text width=60mm}{},
		if = {level == 2}{
			outer color=lightgreen,
			for tree = {
				text width=50mm,
				child anchor=west
			}
		}{},
		if = {level == 3}{outer color=lightpink, text width=45mm}{}
	}
	[Destilador solar, outer color=lilac,
	for tree = {
		for children = {
			grow'=0,
			folder
		}
	}
		[Activo
			[Integrado
				[Colector parabólico]
				[Colector de placa plana]
				[Colector tubular]
				[Colector cilíndrico]
				[Bomba de calor integrada]
			]
			[Híbrido
				[Integrado a un estanque solar]
				[Integrado a módulos fotovoltaicos]
				[Integrado a un extractor]
				[Potenciado por calor residual]
			]
		]
		[Pasivo
			[Con forma de V]
			[De una vertiente o inclinado]
			[Dos vertientes o de tejado]
			[Estructura de cascada]
			[Esférico]
			[Hemisférico]
			[Multi-efecto]
			[Pirámide triangular]
			[Con condensador]
			[Destilador tubular]
			[Con reflector]
			[Cóncavo]
		]
	]
\end{forest}

			\caption{Subclasificaciones de la \acrlong{ds}}
			\floatfoot{Figura adaptada de \cite{singh_active_2020}}
			\label{fig:destilación-solar}
		\end{figure}
		
		