\section{Irradiación solar}

	\subsection{Clasificaciones de la irradiancia}
			
		La irradiancia que recibe un objeto es clasificada de acuerdo a las interacciones por las que ese haz de luz pasó antes de llegar a dicho objeto.
		
		\begin{itemize}
			\item \textbf{Radiación directa:} Aquella que no tuvo interacción con otros cuerpos y que llega sin cambio de dirección.
			\item \textbf{Radiación difusa:} Aquella que sufrió algún choque con un cuerpo de la atmósfera, por ejemplo, la luz que atravesó una nube y fue difractada.
			\item \textbf{Radiación reflejada:} Aquella que proviene de la reflexión de la radiación directa.
		\end{itemize}
	
	\subsection{Propiedades espectrales de la radiación solar}
		
		Todas las radiaciones electromagnéticas viajan en el vacío con la misma rapidez ($\gls{c} = \qty{2.99792458e8}{\m\per\s}$) sin importar la fuente de emisión \cite{young_fisica_2009}, y todos los cuerpos que poseen temperatura por arriba de los 0 grados kelvin emiten radiación electromagnética derivada del movimiento vibracional de sus moléculas; las frecuencias emitidas y asociadas a esta vibración dan origen a la radiación térmica. Nótese que la luz no necesariamente tiene la misma energía a pesar de viajar a la misma rapidez.
		
		Cerca del \percent{99} de la potencia recibida por el sol está entre los \qtyrange{300}{2500}{\nm}. En las~\cref{fig:Espectro-radiación-solar,fig:Potencia-solar} notamos que de la irradiancia solar recibida, un aproximado del \percent{44} de la energía total corresponde al espectro visible mientras que un \percent{52} pertenece a la región del infrarrojo y el \percent{4} restante correspondería a la radiación ultravioleta \cite{weinstein_spectral_nodate}.
		
		\begin{figure}[H]
			\centering
			\begin{subfigure}{\linewidth}
				\centering
				\includegraphics[width=\linewidth]{Marco-teórico/Espectro-radiación-solar.png}
				\caption{Espectro de radiación solar incidente sobre la Tierra}
				\label{fig:Espectro-radiación-solar}
			\end{subfigure}
			\begin{subfigure}{\linewidth}
				\centering
				\includegraphics[width=\linewidth]{Marco-teórico/Potencia-solar.png}
				\caption{Potencia recibida del sol en la Tierra}
				\label{fig:Potencia-solar}
			\end{subfigure}
			\caption{Imágenes traducidas de \cite{weinstein_spectral_nodate}}
			\label{fig:solar-spectrum-brilliant}
		\end{figure}
		
%		Las longitudes de onda entre \qtyrange{700}{1e7}{\nm} son asociadas directamente al calor 
		
	
	\subsection{Irradiancia solar en la Tierra}\label{sec:Irradiancia-Tierra}
	
		El sol posee una temperatura superficial aproximada de \SI{5778}{\kelvin}, la cual emite calor en forma de radiación constantemente, de la cual, en el tope de la atmósfera terrestre se recibe un valor conocido como constante solar (\gls{G}) cuyo valor actualizado es de \SI{1360.8}{\watt\per\m\tothe{2}} $\pm$ \SI{0.5}{\watt\per\m\tothe{2}}. Sin embargo, no toda esa energía llega a la superficie terrestre~\cref{fig:atenuación-radiación-solar} debido a una serie de factores como el movimiento de rotación y traslación así como la inclinación de la Tierra, las condiciones climáticas y geográficas propias de la zona, del día  y de las interacciones a lo largo de su trayectoria por la atmósfera \cite{garcia_valladares_aplicaciones_2017}.
		
		\begin{figure}[H]
			\centering
			\includegraphics[width=\linewidth]{Marco-teórico/atenuación-radiación-solar.png}
			\caption{Atenuación de la radiación solar por su paso por la atmósfera}
			\label{fig:atenuación-radiación-solar}
			\floatfoot{Imagen obtenida de \cite{garcia_valladares_aplicaciones_2017}}
		\end{figure}
		
		\subsubsection{Variación con respecto a la latitud}
			
			Como se mencionó anteriormente, el recurso solar varía de acuerdo a la latitud (\gls{latitud}). Ignorando la inclinación de la Tierra, esta variación viene dada por~\eqref{equ:irradiancia-latitud}.
			
			\begin{equation}
				\label{equ:irradiancia-latitud}
				I_{\gls{latitud}} = I_{\text{Superficial}} \times \dfrac{\cos{\gls{latitud}}}{\pi}
			\end{equation}
			
			Esta ecuación deriva de la razón entre el área proyectada y el área superficial verdadera. La Tierra se proyecta como un círculo al recibir la luz solar, sin embargo, en realidad se trata de una vista del área de una esfera. Si tomamos una franja infinitesimal de tamaño $d\phi$, podemos crear los dos rectángulos observados en la~\cref{fig:irradiancia-latitud} donde $r_{\text{E}} $ es el radio de la Tierra.
			
			\begin{figure}
				\centering
				\includegraphics[
					width = \linewidth,
					height = 50mm,
					keepaspectratio
				]{Marco-teórico/irradiancia-latitud.png}
				\caption{Irradiancia solar a latitud $\phi$}
				\floatfoot{Imagen traducida de \cite{weinstein_solar_nodate}}
				\label{fig:irradiancia-latitud}
			\end{figure}
			
			Como se observa, intensidad de la radiación solar cae rápidamente y a pesar de ser una cantidad enorme de energía, esta se encuentra dispersa en toda el área superficial real.
		
		\subsubsection{Direccionalidad de la radiación solar}
			
			El Sol se proyecta en el cielo como un círculo, no como un punto, haciendo que sea una fuente de luz imperfecta cuyos rayos no son perfectamente colimados ya que vienen con una ligera desviación angular que se puede aproximar mediante \eqref{equ:half-divergence-angle}.
			
			\begin{equation}\label{equ:half-divergence-angle}
				\gls{half-angle-sun} = \arctan\left(\dfrac{\gls{r-sun}}{\gls{d-sun-earth}}\right)
			\end{equation}
			
			Usando~\eqref{equ:half-divergence-angle} se calcula que el ángulo medio de radiación solar es $\gls{half-angle-sun} = \ang{0.275}$.
		
	\subsection{Potencial solar en México}

			México se encuentra en una zona geográfica de alta irradiación solar, como se aprecia en la~\cref{fig:radiacion-solar-promedio-mexico}, diariamente se recibe un promedio mínimo de \SI{4.4}{\kWh\per\m\tothe{2}} y un máximo de \SI{6.3}{\kWh\per\m\tothe{2}} según la zona geográfica, lo que lo sitúa como un país con alta capacidad en energía solar.

			\begin{figure}[H]
			\centering
			\begin{subfigure}[b]{0.45\textwidth}
				\centering
				\includegraphics[width=\linewidth,height=50mm, keepaspectratio]{Marco-teórico/radiación-solar-promedio-mundial.png}
				\caption{Radiación solar promedio en el mundo}
				\label{fig:radiacion-solar-promedio-mundial}
			\end{subfigure}
			\hfill
			\begin{subfigure}[b]{0.45\textwidth}
				\centering
				\includegraphics[width=\linewidth,height=50mm, keepaspectratio]{Marco-teórico/radiación-solar-promedio-mexico.png}
				\caption{Radiación solar promedio en México}
				\label{fig:radiacion-solar-promedio-mexico}
			\end{subfigure}
			\caption{Radiación solar}
			\floatfoot{Imágenes obtenidas de \cite{garcia_valladares_aplicaciones_2017}}
			\label{fig:radiacion-solar}
		\end{figure}