\section{Corrosión}

		La \acrfull{ampp} explica que la corrosión es un fenómeno natural comúnmente definido como el deterioro de un material, por lo usual metálico, resultante de una reacción química o electroquímica con el ambiente. La corrosión puede causar daños peligrosos y costosos a cualquier sistema que incluya metal, entre ellos estructuras metálicas y tuberías, afortunadamente, existen diversas formas de prevenir o controlar la corrosión y sus impactos \cite{ampp_what_nodate}.

		A excepción de algunos tipos de corrosión debido a altas temperaturas, todas se producen a través de la acción de la célula electroquímica, la cual está compuesta habitualmente por:

		\begin{itemize}[columns=2]
			\item Un ánodo donde la oxidación y pérdida de metal ocurren
			\item Un cátodo donde se producen efectos de reducción y protección
			\item Un conjunto de trayectorias metálicas y electrolíticas entre el ánodo y el cátodo por los que fluye la corriente electrónica e iónica
			\item Una diferencia de potencial que activa la célula
		\end{itemize}

		\subsection{Formas de corrosión}

			\subsection{Clasificación por su forma de identificación}

				\inputtcb{Corrosion}

			\subsubsection{Corrosión general o corrosión uniforme}

				Este tipo de corrosión sucede de manera uniforme sobre una superficie, al encontrase al rededor de toda la superficie, se produce un adelgazamiento casi uniforme que se reconoce por la rugosidad del material y productos de la corrosión como el óxido.

				Este mecanismo de corrosión típicamente sucede por la interacción con el ambiente (electrolito), el cual genera un ataque electroquímico sobre la superficie del material. Las diferencias en la composición u orientación entre pequeñas áreas en la superficie del metal crean ánodos y cátodos que facilitan el proceso de corrosión \cites{rodriguez_suarez_industria_2017}{ampp_what_nodate}.

			\subsubsection{Corrosión localizada}

				A diferencia de la corrosión general, este tipo ataca lugares discretos del sistema, siendo más peligrosa, pues al tener área reducida, es más difícil de controlar; además, este tipo de corrosión es la que genera mayor daño a las tuberías de revestimiento en ambientes con \ch{H2S} o \ch{CO2}.

		\subsection{Métodos para mitigar la corrosión}
			
			\textit{Selección de materiales}\par
			Considerar diversos factores al momento de elegir los materiales es esencial para disminuir en lo más posible la corrosión, entre ellos, se encuentran aunque no limitados a: condiciones ambientales y de operación, tipo de producto con el que interactuará el material, tiempo de vida estimado y proximidad a fenómenos de corrosión.
			
			\textit{Planeación y ejecución de medidas preventivas}\par
			Algunas medidas que se pueden tomar mitigar los impactos de la corrosión son: aplicación de recubrimientos protectores; medición e inspección de la corrosión para llevar un control; uso de protección catódica; uso de inhibidores químicos y la creación de un plan de manejo para la corrosión.