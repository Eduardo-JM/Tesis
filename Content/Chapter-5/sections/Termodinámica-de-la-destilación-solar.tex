 \section{Termodinámica de la destilación solar}
	
	\subsection{Aumento de temperatura}
		
		La energía térmica o calor así mismo como el sonido, es el movimiento vibracional de los átomos y moléculas. Las vibraciones de baja frecuencia corresponden al sonido mientras que frecuencias más altas se expresan en forma de calor \cite{chandler_explained_2010}. A la energía asociada a esta excitación se le conoce como fonón.
		
		La cantidad de energía térmica en un cuerpo definirá la temperatura de este y su incremento se puede cuantificar a través de~\eqref{equ:incremento-temperatura}
		
		\begin{equation}
			\label{equ:incremento-temperatura}
			\Delta\gls{T} = \dfrac{\gls{Q}}{\gls{cs}\gls{m}}
		\end{equation}
	
	\subsection{Pérdidas de calor}
		
		\subsubsection{Convección}
			dsfdsf
		
%		A diferencia de los fotones, los fonones interactúan entre sí incluso si tienen diferentes longitudes de onda, por lo que su interacción se vuelve más caótica