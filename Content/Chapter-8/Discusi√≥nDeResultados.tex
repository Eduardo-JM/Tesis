\chapter{Discusión de resultados}

	A pesar de haber tenido un avance significativo en el desarrollo de este trabajo cumpliendo 8 de 9 objetivos, la evaluación del desempeño del desalinizador no se pudo concretar debido a factores no contemplados durante el cronograma de actividades, pues aunque se sortearon algunos riesgos previstos, los siguientes factores retrasaron las actividades planeadas:
	
	\begin{itemize}
		\item En el diseño se concibieron coples impresos en resina 3D para integrar el ventilador a los módulos debido a que las medidas no son comerciales; sin embargo, se tuvieron que evaluar diferentes resinas ya que las geometrías impresas tenían defectos que impedían su uso, por ejemplo, exceso de resina curada obstruyendo los ductos o impidiendo un ensamble correcto.
		\item Negación del uso de maquinaria disponible en los talleres de manufactura y necesidad de ir a instalaciones externas para completar la manufactura de los módulos.
		\item Se enfrentó a la problemática de clasificaciones erróneas de productos por parte de algunas tiendas proveedoras, lo que resultó en la entrega incorrecta de componentes. Esto generó la necesidad de cambiar las piezas varias veces, ocasionando retrasos adicionales en el proceso.
	\end{itemize}
	
	\subsection{Aportes a la investigación}
	
		La revisión exhaustiva de la literatura destaca el valor teórico que el presente trabajo aporta, dado que los antecedentes que abordan las problemáticas que atiende el diseño propuesto son realmente escazos. En este contexto, se crean nuevas oportunidades de diseño que permitan aumentar significativamente la competitividad de la destilación solar activa.
		
		En cuanto al desempeño energético, este trabajo acentúa la necesidad de implementar estrategias que consideren y optimicen el uso de la energía en los destiladores solares. Al ser un método térmico, una gestión eficiente del calor residual puede desempeñar un papel valioso en su rendimiento. Desde esta perspectiva, este trabajo sugiere un amplio espacio para la innovación en el diseño de destiladores solares con la incorporación de nuevas tecnologías y materiales. 
		
		En este diseño en particular, se propone la utilización de arena sílica como una especie de batería térmica para suavizar las curvas de producción, contrarrestando así la intermitencia y variación de la energía solar. No obstante, a medida que la investigación en la gestión térmica y materiales se extiende, las consideraciones tomadas para este diseño se pueden expandir en numerosas direcciones. Se pueden explorar recubrimientos con mayor absortividad en el rango deseado, materiales que proporcionen un mejor aislamiento térmico, superficies de condensación más eficientes, entre otras posibilidades.
	
	\subsection{Desafíos y recomendaciones a futuro}
		
		\subsubsection{Manufacturabilidad y costos}
			
			La manufacturabilidad de piezas presentó un desafío importante, el costo de los componentes es alto, esto se entiende por ser un prototipo, sin embargo, la escalabilidad de un sistema así debe tener esto en consideración. Para dar una idea, el intercambiador de calor propuesto fue cotizado en poco más de 3 mil pesos mexicanos a la fecha, por lo que estrategias para reducir costos se hacen indispensables. La creación de moldes podrían ser una alternativa viable si se pretendiera escalar el sistema.
		
			En cuanto a las impresiones 3D se sugiere mantener este método de manufactura aditiva ya que permite resultados precisos y se pueden fabricar \textit{in situ}, sin embargo, se tiene que tener a consideración la compatibilidad entre las resinas y las impresoras, ya que aunque el fabricante advierta que son altamente compatibles, factores ambientales o de calibración pueden causar problemas durante la impresión 3D.
			
		
		\subsubsection{Mejoras en el diseño}
		
			El diseño propuesto mostró buenos resultados en las simulaciones de acuerdo a lo calculado inicialmente, sin embargo, se identificó que el mecanismo de evaporación tiene oportunidad de mejora, pues como se discutió con anterioridad, se incorporó un diseño impreso en 3D para aumentar la superficie de evaporación; de aquí se distingue que un nuevo y mejor diseño de la cámara de evaporación podría mejorar el rendimiento del desalinizador al permitir una mayor evaporación.
		
		\subsubsection{Mejoras en el sistema de control}
			
			Se distinguen dos puntos de mejora para el modelo de control difuso:
			
			\begin{itemize}
				\item Evaluar el desempeño real de la fuzzyficación de las variables y ajustar los valores previamente establecidos mediante las simulaciones.
				\item Incorporar la variable de la radiación solar en tiempo real al modelo de control difuso permitirá tomar decisiones más acertadas para incrementar, disminuir, comenzar o detener el flujo de agua.
			\end{itemize}
		
		
		En conclusión, los desafíos que surgieron durante el desarrollo de este proyecto han dado las pautas necesarias para futuras mejoras en la manufacturabilidad, el diseño y el sistema de control del desalinizador. La adopción de estrategias efectivas para la reducción de costos, el refinamiento del diseño de la cámara de evaporación y la integración de la radiación solar en el sistema de control son aspectos fundamentales para progresar hacia un desalinizador más eficiente y económicamente viable. Estos avances no solo mejorarán la eficacia del desalinizador, sino que también contribuirán a su rentabilidad y sostenibilidad a largo plazo.

			
			