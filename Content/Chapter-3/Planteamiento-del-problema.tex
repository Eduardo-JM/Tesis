\chapter{Planteamiento del problema}

	Globalmente existen más de \num{18000} plantas desalinizadoras que contribuyen a garantizar el derecho de acceso a agua limpia y saneamiento a millones de personas. De acuerdo a las distintas condiciones como lo son el clima, la geografía, la política y la accesibilidad tecnológica, se selecciona el método de desalinización más adecuado, entre ellos, la ósmosis inversa se ha convertido en la tecnología más popular a nivel industrial \cite[11]{lattemann_chapter_2010} y se prevé que siga aumentando su presencia \cite{intelligence_ro_2021} debido a que es un proceso de alto rendimiento y económicamente favorable en relación al costo por litro de agua producido.
	
	Aunque esta industria ha madurado rápidamente en los últimos 40 años, aún existen áreas de oportunidad en los procesos ya que en general, la desalinización se considera de alto consumo energético y de grandes costos de construcción y operación. Aunado a ello, se presentan varios retos ambientales que comprendes entre otros: la huella ecológica de su construcción, las emisiones de \acrshort{gei} productos de la operación, impactos asociados a la obtención del agua salada y la materia prima que se utilice y la disposición final de residuos. Siendo en ocasiones el último facto un criterio que define la viabilidad final de una planta \cite{singh_experimental_2016}.
	
	Dado lo anterior, la incorporación de estrategias para incorporar energía renovable a los procesos de desalinización y el desarrollo de las tecnologías disponibles para aumentar la eficiencia energética resulta en una tarea indispensable para cubrir sosteniblemente la creciente demanda de agua. Está claro que esta visión debe ser complementada por la concienciación de la población y la creación de políticas para un mejor manejo de los recursos hídricos disponibles.
	
	Con base en lo ya expuesto, este proyecto plantea el desarrollo de un destilador solar activo e híbrido capaz de desalinizar a un ritmo lo más constante posible de acuerdo a las condiciones climáticas y geológicas disponibles. Para ello, se plantean los siguientes retos de ingeniería específicos a resolver:
	
	\begin{itemize}
		\item Obtención y caracterización del agua salada a usarse como materia prima.
		\item Obtención y análisis de los datos ambientales del lugar de desarrollo para identificar las variables ambientales de interés que influirán en la operación del sistema propuesto.
		\item Diseño del sistema y elaboración del modelo térmico que lo regirá así como la implementación de estrategias para reducir las pérdidas e intensificar la transferencia de calor.
		\item Propuesta de la capacidad de desalinización del sistema y selección de las lentes de concentración de acuerdo a la potencia requerida.
		\item Monitoreo del sistema y regulación del flujo de agua.
		\item Construcción del sistema.
	\end{itemize}
	
%	analizar la implementación sostenible de destiladores solares activos en lugares de alta irradiación solar con fácil acceso a cuerpos acuosos salobres, pues es un lugar idóneo de aplicación donde esta tecnología consigue explotar a su favor las condiciones climáticas y geográficas.
%	
%	Para ello debemos superar diversos retos de ingeniería específicos a resolver, entre los que se encuentran: