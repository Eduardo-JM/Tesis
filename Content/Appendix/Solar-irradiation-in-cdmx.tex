\chapter{Código de julia para analizar las condiciones de la energía solar térmica en la CDMX}
\label{ch:solar-irradiation-code}

\section*{Paquetes importados}

\begin{lstlisting}
using Pkg
Pkg.add("Plots")
Pkg.add("HTTP")
Pkg.add("CategoricalArrays")
Pkg.add("DataFrames")
Pkg.add("CSV")
Pkg.add("Missings")
Pkg.add("Statistics")

using HTTP
using DataFrames
using CSV
using CategoricalArrays
using Missings
using Statistics
using Plots
\end{lstlisting}

\section*{Declaración de diccionarios}

\begin{lstlisting}
parameters = Dict(
	"hourly" => [
		"PS",
		"WS2M",
		"QV2M",
		"CLRSKY_SFC_SW_DWN",
		"ALLSKY_SFC_SW_DWN",
		"CLOUD_AMT",
		"ALLSKY_SFC_UVA",
		"ALLSKY_SFC_UVB"
	]
);

descriptions = Dict(
	"PS" => [
		"Surface Pressure",
		"The average of surface pressure at the surface of the earth.",
		"kPa"
	],
	"WS2M" => [
		"Wind Speed at 2 Meters",
		"The average of wind speed at 2 meters above the surface of the earth.",
		"m/s"
	],
	"QV2M" => [
		"Specific Humidity at 2 Meters",
		"The ratio of the mass of water vapor to the total mass of air at 2 meters (kg water/kg total air).",
		"g/kg"
	],
	"CLRSKY_SFC_SW_DWN" => [
		"Clear Sky Surface Shortwave Downward Irradiance",
		"""The total solar irradiance incident (direct plus diffuse) on a horizontal plane at the surface of the earth under clear sky conditions. An alternative term for the total solar irradiance is the "Global Horizontal Irradiance" or GHI.""",
		"Whr/m^2"
	],
	"ALLSKY_SFC_SW_DWN" => [
		"All Sky Surface Shortwave Downward Irradiance",
		"""The total solar irradiance incident (direct plus diffuse) on a horizontal plane at the surface of the earth under all sky conditions. An alternative term for the total solar irradiance is the "Global Horizontal Irradiance" or GHI.""",
		"Whr/m^2"
	],
	"CLOUD_AMT" => [
		"Cloud Amount",
		"The average percent of cloud amount during the temporal period.",
		"%"
	],
	"ALLSKY_SFC_UVA" => [
		"All Sky Surface UVA Irradiance",
		"The ultraviolet A (UVA 315nm-400nm) irradiance under all sky conditions.",
		"W/m^2"
	],
	"ALLSKY_SFC_UVB" => [
		"All Sky Surface UVB Irradiance",
		"The ultraviolet B (UVB 280nm-315nm) irradiance under all sky conditions.",
		"W/m^2"
	]
);

months = Dict(
	1 => "Enero",
	2 => "Febrero",
	3 => "Marzo",
	4 => "Abril",
	5 => "Mayo",
	6 => "Junio",
	7 => "Julio",
	8 => "Agosto",
	9 => "Septiembre",
	10 => "Octubre",
	11 => "Noviembre",
	12 => "Diciembre"
);
\end{lstlisting}

\section*{Constantes para la construcción de la URL}

\begin{lstlisting}
const LATITUDE::Float64 = 19.3038;
const LONGITUDE::Float64 = -99.0732;
START_DATE::Int64 = 20180101;
END_DATE::Int64 = 20221231;
\end{lstlisting}

\section*{Constantes}

\begin{lstlisting}
START_HOUR::Int64 = 4;
END_HOUR::Int64 = 18;
\end{lstlisting}

\section*{Solicitud de datos a la base de datos Power Larc de la NASA}

\begin{lstlisting}
path = joinpath(pwd(), "files");
if !isdir(path)
	mkdir(path)
end
for key in keys(parameters)
	URL::String = string(
		"https://power.larc.nasa.gov/api/temporal/$key/point",
		"?parameters=",
		join(parameters[key], ","),
		"&community=RE",
		"&longitude=$LONGITUDE",
		"&latitude=$LATITUDE",
		"&start=$START_DATE",
		"&end=$END_DATE",
		"&format=CSV"
	);
	download(URL, joinpath(path, "$START_DATE-$END_DATE-$key.csv"))
end
\end{lstlisting}

\section*{Tratamiento y análisis de datos}

Nota: Se confía en la fuente, por lo que se asume que no existen valores nulos en el DataFrame. Sin embargo, se observa que hay valores para rellenar iguales a -999. Se observa que estos valores son consecutivos a partir del 01/04/2022. Se espera que sea debido al periodo de recolección que pasa de ser cada hora a diario. Se buscará entonces otro dataset. Se encontró que en efecto, desde esa fecha no hay datos, aunque al corregirlo a diario, se reduce de 5 a 3 columnas la falta de datos, aunque aún se observan algunos datos faltantes.

\begin{lstlisting}
na_val = -999;
min_year::Int64 = trunc(Int, (START_DATE/10000));
max_year::Int64 = trunc(Int, (END_DATE/10000));
graphs_path = joinpath(pwd(), "graphs");
if !isdir(graphs_path)
	mkdir(graphs_path)
end

for file in readdir(path)
	df = DataFrame(CSV.File(joinpath(path, file), header=17))
	subset!(
		df,
		names(df) .=> ByRow(x -> x != na_val),
		:HR => ByRow(x -> x >= START_HOUR && x <= END_HOUR),
		:YEAR => ByRow(x -> x >= min_year)
	)

	if (isequal(file[19:end-4], "hourly"))
		gd = groupby(df, [:HR, :MO, :YEAR])
	else
		gd = groupby(df, [:MO, :YEAR])
	end

	mean_dg = combine(gd, parameters[file[19:end-4]] .=> mean)
	println(describe(mean_dg))

	for description in keys(descriptions)
		for hour in START_HOUR : END_HOUR
			plot_data = filter(row -> row.HR === hour, mean_dg)
			figure = plot(
				[months[x] for x in minimum(plot_data.MO) : maximum(plot_data.MO)],
				[
					filter(row -> row.YEAR === year, plot_data)[:, string(description, "_mean")]
						for year in min_year : max_year -1
				],
				title = "$(descriptions[description][1]) at $hour",
				label = ["2018" "2019" "2020" "2021"],
				ylabel = "[$(descriptions[description][3])]",
				xlabel = "Mes",
				xrotation = 90
			);
			savefig(joinpath(graphs_path, string(description, "_", hour, ".png")))
		end
	end
	for description in keys(descriptions)
		figure = plot();
		for year in min_year : max_year -1
			plot_data = filter(row -> row.YEAR === year, mean_dg)
			figure = plot!(
				[plot_data.HR],
				[months[x] for x in plot_data.MO],
				plot_data[:, string(description, "_mean")],
				title = "$(descriptions[description][1])",
				label = "$year",
				zlabel = "[$(descriptions[description][3])]",
				xlabel = "Hora",
				ylabel = "Mes",
				yrotation = 90
			);
		end
		savefig(joinpath(graphs_path, string(description, "_3d.png")))
	end
	
	md = groupby(df, [:DY, :MO, :YEAR])
	irradiation_keys = filter(key -> occursin("SFC", key), keys(descriptions))
	sum_md = combine(md, irradiation_keys .=> sum)

	for key in irradiation_keys
		for year in min_year : max_year -1
		    data2plot = filter(row -> row.YEAR === year, sum_md)
			surface(
				[months[x] for x in data2plot.MO],
				[data2plot.DY],
				data2plot[:, string(key, "_sum")],
				title = "$(descriptions[key][1]) in $year",
				#label = "$year",
				zlabel = "[$(descriptions[key][3]) día]",
				xlabel = "Mes",
				ylabel = "Día",
				xrotation = 90
			)
			savefig(joinpath(graphs_path, string(key, "_surface_", year, "_3d.png")))
		end
	end
end
\end{lstlisting}

