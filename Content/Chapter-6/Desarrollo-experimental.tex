\chapter{Desarrollo experimental}
	
	Este trabajo busca realizar una investigación cuasi-experimental para reportar el comportamiento  del destilador solar propuesto de acuerdo a la influencia de las variables independientes descritas en la~\cref{table:variables-independientes-desarrollo-experimental}.
	
	\begin{longtblr}[
		caption = {Variables del desarrollo experimental},
		label = {table:variables-independientes-desarrollo-experimental},
		note{*} = {Se puede ejercer control directo sobre esta variable.}
	]{
		colspec = {X[l] c c X[2, l]},
		hlines,
		vlines,
		row{odd} = {bg=tablerowblue},
		row{1} = {
			bg = tabletitleblue,
			fg=white,
			font = \bfseries,
			halign=c
		},
		rowhead = 1,
		rows={m}
	}
		Variable & Clasificación & Tipo de dato & Influencia en el modelo\\
		Temperatura de entrada del agua\TblrNote{*}
			& Independiente
			& Cuantitativa
			& Establece la temperatura inicial del modelo térmico para la ebullición del agua\\
		Presión atmosférica
			& Independiente
			& Cuantitativa
			& Modifica la temperatura a la cual sucede el cambio de fase del agua\\
		Temperatura ambiente
			& Independiente
			& Cuantitativa
			& Influye en las pérdidas de calor que tendrá el sistema\\
		Clima (nubosidad)
			& Independiente
			& Cualitativa
			& Punto de comparación rápido para ver la influencia de la nubosidad sobre el desempeño\\
		Velocidad de viento
			& Independiente
			& Cuantitativa
			& Se asocia directamente a las pérdidas de calor por convección\\
		Hola del día
			& Independiente
			& Cuantitativa
			& Determina la posición angular del Sol y se relaciona con la irradiación solar\\
		Irradiación solar promedio
			& Independiente
			& Cuantitativa
			& Determina la potencia solar recibida en nuestro modelo de transferencia de calor\\
		Temperatura de salida del agua
			& Dependiente
			& Cuantitativa
			& Asociada al cambio de fase del agua\\
		Temperatura del recibidor solar
			& Dependiente
			& Cuantitativa
			& Influye en los modelos térmicos para la destilación del agua\\
		Caudal de agua\TblrNote{*} 
			& Dependiente
			& Cuantitativa
			& Influye en los modelos térmicos y la tasa de desalinización\\
		Propiedades físico-químicas del agua
			& Dependiente
			& Cuantitativa
			& Indica la calidad del agua
	\end{longtblr}
	
	\section{Grupo de control}
		
		Con base en las variables observadas en la~\cref{table:variables-independientes-desarrollo-experimental} se distinguieron los grupos de control descritos en las sub-secciones siguientes.
		
		\subsection{Agua}
			
			Las muestras de agua se seleccionaron guiándose en la~\cref{table:clasificacion-agua-tds}. Para ello, la obtención de agua de mar se simula con sales marinas para acuario y se proponen 3 grupos de control descritos en la~\cref{table:grupo-control-agua} para evaluar los casos límite y promedio de la salinidad del agua de mar.
			
			\begin{longtblr}[
				caption = {Grupo de control del agua de mar},
				label = {table:grupo-control-agua}
			]{
				colspec = {X[c] X[2, c]},
				hlines,
				vlines,
				width = 0.5\linewidth,
				row{odd} = {bg=tablerowblue},
				row{1} = {
					bg = tabletitleblue,
					fg=white,
					font = \bfseries,
					halign=c
				},
				rows={m}
			}
				Muestra & Salinidad (\unit{\mg\per\litre})\\
				1 & \num{30000}\\
				2 & \num{35000}\\
				3 & \num{40000}
			\end{longtblr}
		
		\subsection{Lugar físico de experimentación}
			
			Debido al alcance del proyecto, se acotó el lugar físico de experimentación a la Unidad Profesional Interdisciplinaria en Ingeniería y Tecnologías Avanzadas ubicada en la Ciudad de México.
			
			\begin{longtblr}[
				caption = {Grupo de control del agua de mar},
				label = {table:grupo-control-fisico}
			]{
				colspec = {X[c] *{3}{c}},
				hlines,
				vlines,
				width = 0.8\linewidth,
				row{odd} = {bg=tablerowblue},
				row{1} = {
					bg = tabletitleblue,
					fg=white,
					font = \bfseries,
					halign=c
				},
				rows={m}
			}
				Zona & Longitud & Latitud & Altitud\\
				Ciudad de México, México
					& \ang{-99;07;32}
					& \ang{19;30;38}
					& \qty{2241}{\m}
			\end{longtblr}