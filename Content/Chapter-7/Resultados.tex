\chapter{Resultados}
	
	\section{Diseño del sistema desalinizador}
		
		Después de un profundo análisis se logró un diseño modular vertical cuyos componentes principales se pueden observar en el~\cref{ch:dibujos-isométricos}.
	
		\subsection{Contenedor de agua de mar}
			
			\begin{figure}
				\centering
				\includegraphics[
					width=\linewidth,
					keepaspectratio
				]{dsf}
			\end{figure}
%		
%		Conductividad térmica
%		
%		Para conocer $\gls{cs}_{\text{agua}}$ de la \cref{table:Conductividad-térmica-agua-salada} se usa \eqref{equ:interpolación-lineal-simple} para obtener la conductividad estimada a los \qty{13}{\degreeCelsius} y a los \qty{94.7}{\degreeCelsius}.
%		
%		\begin{align*}
%			\gls{cs}_{13} &= \qty{0.586}{\watt\per\m\kelvin} + \dfrac{\qty{0.601}{\watt\per\m\kelvin} - \qty{0.586}{\watt\per\m\kelvin}}{\qty{20}{\degreeCelsius}-\qty{10}{\degreeCelsius}} \times (\qty{20}{\degreeCelsius} - \qty{13}{\degreeCelsius})\\
%			\gls{cs}_{13} &= \qty{0.591}{\watt\per\m\kelvin}\\
%			\gls{cs}_{94.7} &= \qty{0.670}{\watt\per\m\kelvin} + \dfrac{\qty{0.674}{\watt\per\m\kelvin} - \qty{0.670}{\watt\per\m\kelvin}}{\qty{100}{\degreeCelsius}-\qty{90}{\degreeCelsius}} \times (\qty{100}{\degreeCelsius} - \qty{94.7}{\degreeCelsius})\\
%			\gls{cs}_{94.7} &= \qty{0.672}{\watt\per\m\kelvin}
%		\end{align*}
%		
%		Una vez obtenidos sacamos el promedio incluyendo el resto del intervalo dándonos que:
%		\begin{equation*}
%			\gls{cs}_{\text{agua}} = \qty{0.638}{\watt\per\m\kelvin}
%		\end{equation*}