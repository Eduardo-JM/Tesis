\begin{abstract}
	\addcontentsline{toc}{section}{Resumen}
	
	\noindent El siguiente trabajo desarrolla una propuesta para la desalinización térmica de agua salada por \gls{destilacion_solar} activa mediante el uso de concentradores solares de fresnel. Ya que la energía solar es una fuente de energía primaria intermitente, se plantea un sistema alimentado de acuerdo a la energía que se logra captar.
	
	\noindent Con el objetivo de conseguir un desempeño energético competente se estableció una serie de análisis térmicos, ópticos y estructurales a seguir que en conjunto de modelos matemáticos y de investigación documental encaminan el diseño del sistema para coadyuvar al derecho de acceso al agua limpia a un precio asequible siendo el fin último contribuir al objetivo de desarrollo sostenible 6.a.
	
	\keywords{desalinización térmica, concentradores solares, destilación solar activa, lentes fresnel}
\end{abstract}

\bgroup
	\selectlanguage{english}
	\begin{abstract}
		\noindent This work bears a technical proposal for thermal desalination of saltwater by active solar distillation through fresnel solar concentrators. Since solar energy is an intermitent primary energy source, it is proposed a system fed in function of the energy that is captured.
				
		\noindent Aiming a proficient energy performance it was established a series of thermal, optical and structural analysis which jointly with mathematical models and documentary reasearch seek to direct the design assisting the right of access to clean water at an affordable price, being the ultimate goal to contribute to the sustainable development goal 6.a.
		
		\keywords[Index terms]{thermal desalination, solar concentrators, active solar distillation, fresnel lenses}
	\end{abstract}
\egroup