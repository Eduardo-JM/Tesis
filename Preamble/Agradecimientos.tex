\chapter*{Agradecimientos}
	
	\vspace*{5mm}
	
	El Instituto Politécnico Nacional me ha dado grandes oportunidades para desarrollarme como persona y como profesionista, dando el espacio y permitiendo que conociera a gente maravillosa que ha dejado huella en mi vida; por eso mismo quiero expresar mi agradecimiento a mi institución que tantas herramientas me ha dado, así mismo, dedico este espacio para reconocer el apoyo y el conocimiento que me ha brindado la Unidad Profesional Interdisciplinaria en Ingeniería y Tecnologías Avanzadas y el Centro de Estudios Científicos y Tecnológicos No. 9 ``Juan De Dios Bátiz''.
	
	A mi familia, que siempre me ha dado su apoyo incondicional y me ha guiado y proveído de las herramientas para crecer como persona de bien y como profesional; a ellos, les doy mi más sincera gratitud. Y por animarme en tiempos de duda y darme fuerza para continuar con este proyecto, por esto y más, mi corazón está lleno de gratitud hacia ellos.
	
	Al Dr. Diego Alonso Flores Hernández, el Dr. Sergio Isai Palomino Resendiz y el Dr. Helvio Ricardo Mollinedo Ponce de León les agradezco enormemente por haberme guiado durante el desarrollo de esta tesis, por haber dado su consejo y su tiempo cada vez que se los requería, y les doy gracias por su inmensa contribución a mi crecimiento académico y profesional.
	
	Al técnico profesionista Héctor Cruz Martín, al Ing. Gustavo Zamudio Rodríguez y a los docentes y técnicos del Laboratorio de Pesados de la UPIITA les agradezco con un grande aprecio por haberme apoyado durante la etapa de manufactura del proyecto.
	
	A mis amigos, Bernardo Alberto Vargas Vidal, José Ismael Gamiño Barocio, Alan Dario Sánchez Cárdenas y Benjamín Guzmán, les agradezco a cada uno por su apoyo a lo largo de la carrera y los aportes que me dieron durante todo este tiempo.
